\section{Forma di stato e linearizzazione}

    Preso il nostro sistema di partenza, vogliamo rappresentarlo nella forma di stato.\\\\
    Per farlo poniamo:\\
    \begin{equation*}
        x=\begin{bmatrix}   
            \omega\\ 
            \dot{\rho}\\ 
            \rho 
        \end{bmatrix}\hspace{0.5cm}
        u=\tau\hspace{0.5cm}
        y=\omega
    \end{equation*}\\
    Per cui possiamo scrivere:
    \begin{equation*}
        \left\{ \begin{array}{ll}
            x_1=\omega\\
            x_2=\dot{\rho}\\
            x_3=\rho
        \end{array} \right.
        \rightarrow
        \left\{ \begin{array}{ll}
            \dot{x_1}=\dot{\omega}\\
            \dot{x_2}=\ddot{\rho}\\
            \dot{x_3}=x_2
        \end{array} \right.
    \end{equation*}
    Sostituendo nelle formule di partenza otteniamo il sistema in forma di stato:
    \begin{equation}
        \left\{ \begin{array}{ll}
            \dot{x_1}= f_1(x,u) = -\dfrac{2x_1x_2}{x_3}-\dfrac{\beta_2x_1}{m}+\dfrac{u}{mx_3}\\\\
            \dot{x_2}= f_2(x,u) = \dfrac{1}{m}\biggl[-\beta_1x_2+m(k-1)\biggl(\dfrac{K_GM}{{x_3}^2}-x_3{x_1}^2\biggr)\biggr]\\\\
            \dot{x_3}= f_3(x,u) = x_2\\\\
            y= h(x,u) = x_1
        \end{array} \right.
    \end{equation}
    Sapendo che \(\rho_e\) è un valore di equilibrio, possiamo trovare la coppia di 
    equilibrio \((x_e,u_e)\)  in modo da poter linearizzare il sistema non lineare nell'equilibrio.\\\\
    Abbiamo trovato la coppia di equilibrio:
    \begin{equation*}
        \Biggl(
            \begin{bmatrix}
                \sqrt{\dfrac{K_GM}{{\rho_e}^3}}\\
                0\\
                \rho_e
            \end{bmatrix},
            \beta_2\sqrt{\dfrac{K_GM}{{\rho_e}}}
        \Biggr)
    \end{equation*}
    \clearpage
    Otteniamo le matrici:\\\\
    \begin{equation}
        \begin{array}{l}
            A=\dfrac{\partial f(x,u)}{\partial x}\Biggl\vert_{\substack{x=x_e\\ u=u_e}}=
            \begin{bmatrix}
                -\dfrac{2x_{2e}}{x_{3e}}-\dfrac{\beta_2}{m} & -\dfrac{2x_{1e}}{x_{3e}} & \dfrac{2x_{2e}x_{1e}}{{x_{3e}}^2}\\\\
                -(k-1)2x_{1e}x_{3e} & -\dfrac{\beta_1}{m}    &   (k-1)\biggl(-2\dfrac{K_GM}{{x_{3e}}^2}-{x_{1e}}^2\biggr)\\\\
                0   &  1    &    0
            \end{bmatrix} \\\\\\
            B=\dfrac{\partial f(x,u)}{\partial u}\Biggl\vert_{\substack{x=x_e\\ u=u_e}}=
            \begin{bmatrix}
                \dfrac{1}{mx_{3e}}\\\\
                0\\\\
                0
            \end{bmatrix}\\\\\\
            C=\dfrac{\partial h(x,u)}{\partial x}\Biggl\vert_{\substack{x=x_e\\ u=u_e}}=
            \begin{bmatrix}
                1&0&0
            \end{bmatrix}\\\\\\
            D=\dfrac{\partial h(x,u)}{\partial u}\Biggl\vert_{\substack{x=x_e\\ u=u_e}}=0
        \end{array}
    \end{equation}

    Da cui ricaviamo il sistema linearizzato

    \begin{equation}
        \left\{ \begin{array}{ll}
            \dot{x_1}= -\dfrac{\beta_2}{m}x_1-\dfrac{2x_{1e}}{x_{3e}}x_2-\dfrac{u_e}{mx^2_{3e}}x_3+\dfrac{u}{mx_{3e}}\\\\
            \dot{x_2}= -[(k-1)2x_{1e}x_{3e}]x_1-\dfrac{\beta_1}{m}x_2+(k-1) \Biggl( -\dfrac{2K_GM}{x^3_{3e}}-x^2_{1e} \Biggr)  x_3\\\\
            \dot{x_3}= x_2\\\\
            y= x_1
        \end{array} \right.
    \end{equation}

